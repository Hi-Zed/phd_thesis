%*******************************************************
% Acknowledgments
%*******************************************************
\pdfbookmark[1]{Acknowledgments}{acknowledgments}

\begin{flushright}{\slshape bip, bop.} \\ \medskip
    --- A robot
\end{flushright}



\bigskip

\begingroup
\let\clearpage\relax
\let\cleardoublepage\relax
\let\cleardoublepage\relax
\chapter*{Acknowledgments}
The PhD itself is a long journey, and it is the culmination of an even longer path. To acknowledge every person that helped me in this long experience, I would need to make this the longest section of this thesis.

Nevertheless, I would like to start from a bit far away, and, first and foremost, thank a person that was pivotal in the ignition of my roboticist career. In my last semester of middle school, I took part in an extra-curriculum activity: cutting and sewing. I do not even remember if we did any actual cutting or sewing, but what I remember clearly was that despite the name, the lab was all about robotics. It was about experimenting, playing and programming a LEGO Mindstorm robot. That is how everything started, that is the event that began a more than fifteen-years long journey in the world of AI and robotics. So, my first ``thank you'' goes to Prof.~Tina Prada, that with an improbable invitation lead me to my first practical experience in robotics.

I want to thank my parents, Giuseppe and Roberta, that supported me during all my studies and kept supporting me after I found my independence, that escalated more quickly than I could possibly imagine. They gave me the little island of peace away from the stress of the PhD where I could do literally nothing, which is, sometimes, more important than doing something. A special thank you to my Grandmother Luigia, for all our lovely telephone chats, and to all the recipe suggestion to have a bit of home far from home. Thanks to my family, because, in the end, we are never truly grown-up in their eyes, and more often than we would like to admit we really need their help and support.

As often happens, my PhD was not a simple and straight path. It was split between two institutions, my alma mater, Politecnico di Milano, and the Knowledge Media Institute, that accepted me first as a visitor and then as a member.

First, I want to thank my supervisor, Matteo Matteucci, for giving me the opportunity to embark in the life-changing experience that is the PhD. Moreover, and most importantly, for giving me the freedom to pursue my own ideas, topics, and challenges, and for never complaining about me spending so much time away. From Politecnico and the AIRLab, I would like to thank all the people I worked with, in particular Giulio Fontana and Luca Bascetta, that shared with me the joys and sorrows of dealing with our beloved robots. All the students I supervised or assisted because I learned more from you than you probably learnt from me, thanks to Andrea, Pietro, Fabio, Carolina, Jordi, and all the robotics class that had the misfortune of being my guinea pigs. A big thank you to all my non-colleagues in Milan. We were such a heterogeneous research group that we rarely talked about our research topics but, no matter what, we always had something to discuss or experiences to share like a big family held together by academic struggles. Thanks to Martino, Francesco, Andrea, Marco, Alessandro, Latta, and Luca.

An special thank you goes to Ilaria, because in a cold dark day during the Swedish winter she offered me a place to do a visiting, but I end up finding a place to stay. By working with her and Manu, I rediscovered a love for robotics that the PhD was slowing draining away. Sometimes it is only by scrambling everything up that you find back the reason why you started. From KMi, I want to thank Enrico Motta, for letting us work on an unconventional topic, and all the people I worked with while juggling my thesis and the SciRoc competition. Thanks to Enrico, Jason, and Ian. In Milton Keynes, I met a lot of amazing people, and I shared with them not only work but also life experiences. As often happens in the academic world, with some of them, I only crossed paths for a few months, but with all of them, I made memories that will stay with me forever. Thanks to all the people I shared lunches, dinners, parties, film nights, hikes, road trips, cooking sessions, barbeques, concerts, weddings, and more. A unique thank you to Thiviyan, Perla, Patrizia, Angelo, Martino, Mano, and Felice.

This section is already long enough, but there is one more person I want to thank. She should appear in many other parts since she is a friend, a colleague, an inspiration, my other half, my partner in crime, my \textit{chef de cusine}. Thank you, Agnese, for all the little things, for every day, and for being always by my side.

\vspace{1cm}

\begin{center}
\rule[1mm]{1cm}{0.4pt}~{\large \textit{\forcesans{Thank you all}}}~\rule[1mm]{1cm}{0.4pt}
\end{center}
\endgroup
