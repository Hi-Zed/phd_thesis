%*******************************************************
% Abstract
%*******************************************************
%\renewcommand{\abstractname}{Abstract}
\pdfbookmark[1]{Abstract}{Abstract}
% \addcontentsline{toc}{chapter}{\tocEntry{Abstract}}
\begingroup
\let\clearpage\relax
\let\cleardoublepage\relax
\let\cleardoublepage\relax

\chapter*{Abstract}
In recent years, robotic applications have surged and robotics popularity has increased both in academia and industry. Researchers of different fields are imagining new ways to intertwine their expertise with robotics, creating challenging and remarkable applications. Companies are developing a new generation of service robots targeted to the general public and meant to be part of our everyday life. Robotics is evolving from its mechatronics roots, more focused on the development of the ideal hardware platform, to explore the advanced functionalities offered by complex applications. We are entering a new software age for robotics. Unfortunately, the tools available to developers are not on par with the expectations. Nowadays, developing an application for a robot is more similar to craftsmanship than engineering. An all-around robotic expert with a combined knowledge about the application, the capability of the platform, and the underlying framework is necessary to guide the design and development process. The objective of this thesis is to provide a collection of methodologies, techniques and tools to support all the actors involved in the development process of a robotic application. Our contribution is threefold, with each part targeted to a specific development role. For the \textit{system designer}, we provide a modelling approach to design, build and analyse the robot architecture, without worrying about the underlying framework. For the \textit{component developer}, we implemented an automatic programming toolchain, which removes the burden of implementing framework-related boilerplate and let the developer focus on the component functionalities. For the \textit{application developer}, we created an abstraction layer on top of the robotic platform, it decouples the robot from its capabilities, creating the equivalent of robot APIs. All these contributions are built for a single purpose but using self-contained technologies, hence they are, at the same time, independent and part of of a continuous design and development process.





\endgroup

\vfill
