%*******************************************************
% Abstract
%*******************************************************
%\renewcommand{\abstractname}{Abstract}
\pdfbookmark[1]{Abstract}{Abstract}
% \addcontentsline{toc}{chapter}{\tocEntry{Abstract}}
\begingroup
\let\clearpage\relax
\let\cleardoublepage\relax
\let\cleardoublepage\relax

\chapter*{Abstract}
This is just a placeholder and memorandum

List of things I did
\begin{itemize}
	\item Part I: modelling and code generation
	\begin{itemize}
		\item AADL based description of robotic architectures
		\item Specialized description of ROS-based components
		\item Reusable models of ROS-based design patterns
		\item ASN.1 and JSON based description of messages and parameters
		\item Engineered reference node for ROS
		\begin{itemize}
			\item Clear separation between middleware and implementation code
			\item Support for internal state machine (node life cycle)
			\item Support for external notification of state
			\item Encapsulated parametrization
			\item Encapsulated internal state
		\end{itemize}
		\item Automatic code generation from the AADL ASN.1/JSON model to the reference ROS node
		\item Applications: full wheelchair architecture, local planner and global planner
	\end{itemize}
	\item Part II: abstraction and capabilities
	\begin{itemize}
		\item Ontology based description of middlewares
		\item Specialization of the description on ROS
		\item Mapping between ROS messages and general robot capabilities
		\item Python based API to interact with the robot using capabilities
		\item JSON based decoupling between ROS and an external interface
		\item Web based interface with the robot exploiting capabilities
	\end{itemize}
	\item Part III: considerations
	\begin{itemize}
		\item Binding between the model and the capabilities
		\item Dependencies between capabilities
		\item Architecture check on the functionalities of the robot
	\end{itemize}
\end{itemize}



\endgroup

\vfill
