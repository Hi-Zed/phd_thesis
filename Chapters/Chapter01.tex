%************************************************
\chapter{Introduction}\label{ch:introduction}
%************************************************
Robotics popularity is growing everywhere. Not only in the academic world, where a lot of research is now applied to robotics, but also in the industrial world, where new company are providing commercial solutions involving robots. Even the general public is now more used to a society where robot coexist and collaborate with humans. The first model of iRobot Roomba was introduced in 2002, this means that today there are teenagers that only know a world where robots as part of the household are the norm.

The current evolution of robotics as a field is similar, in a way, to the growth in popularity of mobile phones, first, and smartphones, later.  Originally, the idea of a personal wireless communication system was only possible in science fiction, then scientific progress and new technology made it possible. At fist only for very few applications, \ie, the military, the railroad system, but later it grew exponentially and today it is part of our everyday life. Many factors made this leap possible: first of all, technological advancements, like miniaturisation, battery life extension, increase in display quality, cheaper computational power, additionally, a sense of need, people felt that a mobile phone was a great addition to their life, lastly, standardization, shared communication platforms, accessible development environments, and multiple abstraction layers.

The same was for robots, originally no more than toys, mechanical puppets and mysterious automata. They existed, as truly autonomous agents, only in the minds and works of writers and directors, and even today we are not able to match those visions. As soon as technology made it possible, the first autonomous arms were developed. Initially applied to heavy industry to replace human in dangerous and highly specialized tasks, later, technical refinements and functionality extensions made them suitable for healthcare and the military. From here it was an explosion of different technologies, shapes and applications. Autonomous arms evolved in precision, power and dexterity,  from the massive industrial arms, to the agile surgical robots.  Soon after the development of the first complex arms, many researchers tried to realize the vision of a full humanoid robot, but, even today, after many progress we are not able to fully replicate the complexity of the human body. Mobile platforms were the next logical step, robots able to autonomously explore and navigate the environment, robots able to reason on what they detect and to react accordingly.

In the last two decades, robotics have been applied in numerous fields and robots assumed a myriad of shapes and functionalities. In industry, robots are used for welding, painting, drilling, cutting, handling dangerous materials, moving heavy objects, pick-and-place, inventory management. In healthcare, today, surgical robots are the norm, but advancement in soft robotics made robots suitable for rehabilitation and elderly care. Most of the recent discoveries of planetary science exist thanks to rovers, autonomous mobile robots that can, unassisted, explore the surface of planets, asteroids and comets. Moreover, maintenance in outer space is extremely dangerous for humans and often impossible, only robotic arm and autonomous probes can perform them. Back on Earth, in our houses and cities, robots are not an unusual sight. There are robot vacuum cleaner and lawn mower,  autonomous robots deliver packages directly to the front door of the house and self-driving public transportation is a reality in various cities. Fully autonomous cars are still only prototypes, however not because of technological limitations, but mostly for economical, social and legal reasons. 
%*****************************************
