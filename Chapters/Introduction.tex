%************************************************
\chapter{Introduction}\label{ch:introduction}
%************************************************

\begin{flushright}{\slshape The story so far: In the beginning the Universe was created. This has made a lot of people very angry and been widely regarded as a bad move.} \\ \medskip
    ---  The Restaurant at the End of the Universe, Douglas Adams 
\end{flushright}

Robotics popularity is growing everywhere. Not only in the academic world, where a lot of research is now applied to robotics, but also in the industrial world, where new companies are providing commercial solutions involving robots. Even the general public is now more used to a society where robot coexist and collaborate with humans. The first model of iRobot Roomba was introduced in 2002, this means that today there are young adults that only know a world where robots in the household are the norm.

The current evolution of robotics as a field is similar, in a way, to the growth in popularity of mobile phones, first, and smartphones, later.  Originally, the idea of a personal wireless communication system was only possible in science fiction, then scientific progress and new technologies made it possible. At fist only for very few applications, \ie, the military, the railroad system, but later it grew exponentially and today it is part of our everyday life. Many factors made this leap possible: first of all, technological advancements, like miniaturisation, battery life extension, increase in display quality, cheaper computational power, additionally, a sense of need, people felt that a mobile phone was a great addition to their life, lastly, standardization, shared communication platforms, accessible development environments, and multiple abstraction layers.

The same was for robots, originally no more than toys, mechanical puppets and mysterious automata. They existed, as truly autonomous agents, only in the minds and works of writers and directors, and even today we are not able to match those visions. As soon as technology made it possible, the first autonomous arms were developed. Initially applied to heavy industry to replace human in dangerous and highly specialized tasks, later, technical refinements and functionality extensions made them suitable for healthcare and the military. From here it was an explosion of different technologies, shapes and applications. Autonomous arms evolved in precision, power and dexterity,  from the massive industrial arms, to the agile surgical robots.  Soon after the development of the first complex arms, many researchers tried to realize the vision of a full humanoid robot, but, even today, after many progress we are not able to fully replicate the complexity of the human body. Mobile platforms were the next logical step, robots able to autonomously explore and navigate the environment, robots able to reason on what they detect and to react accordingly.

In the last two decades, robotics have been applied in numerous fields and robots assumed a myriad of shapes and functionalities. In industry, robots are used for welding, painting, drilling, cutting, handling dangerous materials, moving heavy objects, pick-and-place, inventory management. In healthcare, today, surgical robots are the norm, but advancement in soft robotics made robots suitable for rehabilitation and elderly care. Most of the recent discoveries of planetary science exist thanks to rovers, autonomous mobile robots that can, unassisted, explore the surface of planets, asteroids and comets. Moreover, maintenance in outer space is extremely dangerous for humans and often impossible, only robotic arms and autonomous probes can perform them. Back on Earth, in our houses and cities, robots are not an unusual sight. There are robotic vacuum cleaners and lawn mowers,  autonomous robots deliver packages directly to the front door and self-driving public transportation is a reality in various cities. Fully autonomous cars are still only prototypes, however not because of technological limitations, but mostly for economical, social and legal reasons. Thanks to the recent progresses in human-robot interaction, the sight of a robotic waiter or concierge, while marvellous, is not completely unexpected. Lastly, unmanned autonomous vehicles, \eg, off-road vehicles, drones, boats, submarines,  have been used successfully in search and rescue missions and to operate in dangerous environments or unreachable by humans, like mountain peaks, volcanos, disaster zones, and contaminated areas.

In this brief history of robots, most of the progress and technological advancements seems related to hardware. More responsive motors, more precise and reliable sensors, cheaper electronic and computational power. All these advancements contributed to what is robotics today. However, software has always been one of the main concerns of any roboticist. The implementation, the logic, is what makes the difference between a mechanism and an intelligent robot. Since their inception, robots have spawned a series of software solutions to implement their functionalities. For example, the Stanford Research Institute Problem Solver, better known by it acronym STRIPS, is an automated planner developed for Shakey that became the foundation of modern action languages. Modern robots have software architecture far more complex than Shakey, they coordinate multiple sensors and actuators, moreover they implement different functionalities and are expected to operate in real time.

This is why, more recently, more or less in the last twenty-five years, a lot of efforts in robotic software revolved around the design of a solution to streamline and simplify the development process. The answer was the introduction of robotic middlewares and frameworks and to rely on component-based designs. This approach fits perfectly the necessities of robotics, components encapsulate functionalities and promote reusability, while a pre-defined communication layer frees the developer from the burden of micromanaging the low-level interactions. After the first wave of ad hoc implementations, few frameworks rose in popularity and become standard de-facto for robotic software development. Today, depending on the specific application, a developer can choose various framework or middleware: OROCOS (or its derivation RoCK), for hard real-time application, SmartMDSD, for a more complete and structured development environment, YARP, for a more light-weight and data-centric approach, or ROS, for more extensive support and development freedom.

Middlewares and frameworks have fuelled the progress of robotic systems, creating the current scenario of robot design and development. Hundreds of components are already available to anyone who wants to implement his own robot and experts can setup the most common functionalities (\ie, teleoperation, mapping, indoor localization and navigation) of a new system in a matter of days. However, the learning curve to reach this kind of expertise is quite steep and extending the functionalities of a robot beyond what is currently available requires a considerable effort not strictly related to the new functionality itself.  By doing again the parallel between robotics and smartphones, we are currently in robotics in the same situation developers were before the standardization introduced by Android. Today, an Android developer can bootstrap and deploy a new application on millions of devices in few steps, thanks to abstraction layers that separate the development environment to the underlying hardware and operating system and thanks to advanced design, development and simulation tools. Of course smartphones are not robots, while there is a great variability from one device to another (\eg, screen size, quality and number of cameras, sensors availability, type of mobile network, \etc), they cannot be compared to the incredible range of sensors, actuators, shapes and functionalities that exist in robotics. For this reason, while robotics can aim to achieve the same streamlined development of smartphones, the approach needs to be different.

\section{Motivations}
Middlewares and frameworks created the present development environment of robotics, but current approaches are not suitable any more for a constantly advancing robotic field. The personal experience of an all-around robotic expert still drives robotic software design and development. When developing a new system or application it is expect that a developer has total expertise on the low-level functionalities provided by the underlying framework and the high-level functionalities to be implemented; while this was possible in the past, it is an unsustainable approach today. Not only it is necessary to create a distinction between different roles in the design and development process of a robot, but is is also necessary to provide to these roles the right tools to fulfil  their tasks.

The \textit{system designer} needs tools to outline the architecture of the system and describe the high-level interactions and requirements of components. This can be achieved using a modelling language to describe components and their inner workings in an agnostic way with respect to the underlying framework. This approach, not only provides the right environment for the designer, but it also provides early detection of errors, an architectural overview of the system and system-level reusability. 

The \textit{component developer} should focus only on the implementation of the internal logic and not on the structure of the component itself, since this is the role of the designer. To do so, the component developer needs an environment that abstracts from the framework-related boilerplate code and provides a contained development space. Potentially, the logic implemented should be portable from one component to another, even if they are not based on the same framework, given they share the same design principles. Building on top of the modelling language used by the system designer, it is possible to achieve the ideal development environment by delegating to an automatic code generator most of the boilerplate implementation, and by defining a bounded reference component that can be used by the component developer as a starting point.

The \textit{application developer} implements high-level functionalities, that should be independent from the underlying architecture of the robot. In practice, this means it should exist an abstraction layer between the low-level capabilities provided by one or more components and the high-level applications. There is a plethora of robots, with different configurations and implementations, however it is possible to abstract most of the capabilities independently from the system. An example could be teleoperation: by defining linear and angular velocity of the mobile platform it is possible to control any robot, independently from their physical configuration. Using these general interfaces the application developer should be able to implement high-level functionalities for multiple robots with minimal modifications. In order to achieve this it is necessary to define the concept of capabilities, to identify them in a robot architecture and to provide a framework-independent way to interact with them.

\section{Thesis contributions}
Our proposed approach revolves around two key factors: formalise the design and development of robotic software and streamline the implementation process for the different experts involved. To do so, we developed a collection of standards, tools and techniques, each one focused on a different aspect or phase of the design and development process to assist each role on their specific task, but all interconnected together to benefit one from another.

For the \textit{system designer} we exploited an existing modelling language to create a suitable description for robotic architectures. We relayed on the fact that the most popular middlewares and frameworks adopted a \textit{component-and-connector structure} to create a generalized approach. Since the aim is to cover the entire development process by supporting all the actor involved, we then focused on creating a more specialized description to model ROS-based architectures. The generalized approach already covered the concept of components (\ie, nodes), ports (\ie, publishers, subscribers, service clients and servers) and connections (\ie, topics and services), while the specialized description goes more in details by providing model for messages and the internal functions of nodes. Moreover, we tried to capture some relevant robotic design pattern, both outside the component (\eg, topic multiplexer) and inside (\eg, message relay). The advantages for the system designer are multiple; a model of the complete system gives an architectural overview which is otherwise impossible to achieve before runtime, moreover it is possible to check, before execution, the compatibility of the communication channels, a functionality that is usually unavailable in those frameworks and middlewares that connect the component at start-up time. Additionally, the designer can rely on a library of already existing templates, this makes the design of the system easier and the resulting architecture more robust. Lastly, by basing this work on an existing modelling language we give the designer the opportunity to exploit all the other tools available for the language, few examples are: latency estimation, computational load, hardware allocation and fault propagation

The \textit{component developer} often works together with the \textit{domain expert}. With our work we provide support for both roles and their interaction. From the model created by the \textit{system designer} we provide an automatic code generation to ROS. The target implementation is based on a reference node specifically engineered to minimize the amount of boilerplate code, moreover, it provides additional features that are usually borne by the \textit{component developer}, few examples are: internal life cycle of the node, well defined initialization procedure, encapsulation of parameters and internal state, clear separation between the middleware and implementation. The latter is particularly important for the role of the \textit{domain expert}; their contribution to the functionalities of a robot is fundamental, they provide control software, local and global planning algorithms, robot behaviour, and more. Since they are expert of a specific domain and carrier of specialized and valuable knowledge, they often do not and ideally should not implement the component directly, but to have access to a suitable interface. In our proposed model and automatic code generation approach  a \textit{domain expert} can implement the functionality independently from the component and then embed it in the model, the automatic code generation will include it in the final implementation.

Lastly, for the \textit{application developer} we developed the concept of robot capabilities. We define them as low or medium level functionalities (\eg, directional movement or navigation) and a developer can use them to interact with the robot (\ie, to send commands of varying complexity) and to receive information from the robot (\ie, to read sensor measurements). The capabilities are defined manually by analysing the configuration and functionalities of different types of robots (\ie, mobile platforms, drones and mobile manipulators), but the active capabilities on a running system are extracted automatically by analysing the ROS graph. On top of the concept of capabilities we developed an abstraction layer to decouple the application from the underlying middleware or framework. In our approach we implemented a bridge between the capabilities and, consistently, ROS based system. To do so, we developed a dynamic node that can manage a bidirectional communication with an external system through different communication channels. We provide dynamically defined Python based API where a developer can interact with the robot through capabilities, moreover, we created a set of remote API where JSON messages can be used to trigger capabilities remotely. To test the effectiveness of this approach in simplifying robot development we created a web interface that can be used to create visual algorithms to program a remote robot.

Even if it is not evident at first glance, all these approaches, techniques, standards and tools are all part of a continuous design and development process. The \textit{system designer} uses the modelling tools and templates to define the architecture of the system. He can embed directly the reference to the source code developed by the \textit{domain expert}, and, using properties, even enrich the component with their evoked capability. Through automatic code generation most of the source code is already available with minimal effort, at this point the \textit{component developer} can finalise the implementation by adding anything that cannot be automatically generated, for example special interfaces with the hardware components or specific initialization and shutdown procedures. The result is a robust system where all the components are known, well designed and well implemented, this is the suitable starting point for an \textit{application developer} to exploit safely the abstraction layer defined using robot capabilities.

\section{Thesis outline}
This thesis is divided in eight chapters:
\begin{itemize}
\item Chapter~\ref{ch:Related}
\item Chapter~\ref{ch:Background}
\item Chapter~\ref{ch:Modelling}
\item Chapter~\ref{ch:code-gen}
\item Chapter~\ref{ch:capabilities}
\item Chapter~\ref{ch:Experiment}
\item Chapter~\ref{ch:Conclusions}
\end{itemize}

\section{Publications}
\begin{itemize}
\item \textbf{G. Bardaro}, M. Matteucci \\
Using AADL to model and develop ROS-based robotic application~\cite{bardaro2017using} \\
\textit{International Conference on Robotic Computing (IRC)}, 2017
\item \textbf{G. Bardaro}, A. Semprebon, M. Matteucci \\
AADL for robotics: a general approach for system architecture modeling and code generation~\cite{bardaro2017aadl} \\
\textit{Journal of Software Engineering for Robotics (JOSER)}, 2017
\item I. Tiddi, E. Bastianelli, \textbf{G. Bardaro}, M. d'Aquin, E. Motta \\
An ontology-based approach to improve the accessibility of ROS-based robotic systems~\cite{tiddi2017ontology} \\
\textit{Knowledge Capture Conference (KCap)}, 2018 \\
\item I. Tiddi, E. Bastianelli, \textbf{G. Bardaro}, E Motta \\
A User-friendly Interface to Control ROS Robotic Platforms~\cite{tiddi2018user} \\
\textit{International Semantic Web Conference (ISWC)}, 2018
\item \textbf{G. Bardaro}, A. Semprebon, M. Matteucci \\
A use case in model-based robot development using AADL and ROS~\cite{bardaro2018use} \\
\textit{International Workshop on Robotics Software Engineering (RoSE)}, 2018
\item \textbf{G. Bardaro}, A. Semprebon, A. Chiatti, M. Matteucci \\
From Models to Software Through Automatic Transformations: An AADL to ROS End-to-End Toolchain~\cite{bardaro2019models} \\
\textit{International Conference on Robotic Computing (IRC)}, 2019
\end{itemize}

%*****************************************
