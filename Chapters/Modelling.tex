%************************************************
\chapter[Modelling]{Modelling}\label{ch:Modelling}
%************************************************

\begin{flushright}{\slshape the sciences do not try to explain, they hardly even try to interpret, they mainly make models. By a model is meant a mathematical construct which, with the addition of certain verbal interpretations, describes observed phenomena.} \\ \medskip
    --- John  von Neumann
\end{flushright}

As said by von Neumann, science is about making models. They can be used to simplify a phenomena and make it easier to understand and define, moreover, a model can be used to quantify or visualise reality. Knowledge extracted by the process of modelling can be reused to create a simulation (\ie, another form of modelling) of the system under analysis. For all these reasons, and because is one of the most innate ability of humans, modelling has always been the cornerstone of science, engineering and arts.

Modelling in engineering is an essential tool for design, analysis and simulation, models have different characteristics and take various shapes. A collection of mathematical formulas can be used to describe a physical phenomena (\eg, friction between the ground and the wheels), or the behaviour of a system (\eg, a control system). Differently, a flow chart is a graphical model of an execution process, while pseudo-code is a textual one. A 3D model capture the physical shape of an object and can be used to study the design or the space occupancy.

This chapter presents various modelling techniques that we used to describe multiple aspect of the architecture of a robot. First, an introduction to the  Architecture Analysis \& Design Language (AADL), a description of the concepts behind the language and how it can be used to model complex system. Then, how the language is exploited to model robotic system and more in particular ROS-based architectures.  Lastly, since AADL is not a data modelling language, we present two approaches based on Abstract Syntax Notation One (ASN.1) and JavaScript Object Notation (JSON) to model the data exchanged in the system (\ie, ROS messages) and the internal state of each component.

\minitoc
\newpage

\section{Architecture Analysis \& Design Language}
The Architecture Analysis \& Design Language is a very powerful modelling language designed to capture the architecture of embedded systems by using architectural models that provide a well-defined and semantically rich description of the runtime architecture. This description encompasses multiple aspects of the system: hardware components, to encode the underlying physical layer of the system, software components, to define the runtime behaviour of the architecture, the interaction between them, for example deployment of software on specific hardware and communication between different execution units, and the defining properties of each modelled element, to better characterise any particular system.

 In AADL, components are defined using a dichotomy between specification and implementation. The component type declaration is used to define the category (see Table~\ref{tab:categories}) and the interfaces (\ie, features) of the component; this correspond to a specification sheet that provides a description of the component as a black box. For a specific type it is possible to define multiple component implementation declarations, each of them defines internal structure of the component (\ie, subcomponents and their interactions). This is equivalent as defining multiple blueprints for building a component from its parts, each of them a possible implementation of an already defined specification. To specify even more the characteristics of a component, especially its runtime behaviour, it is possible to use properties. AADL already provides a collection of predefined properties, and more are available by including standard annexes for specific analyses, moreover, an user can defined his own properties by defining additional properties sets. Together, all these declarations (\ie, type with implementation whit a set of properties) define a pattern for a component, which are referred as component classifiers.
 
\begin{table}
    \myfloatalign
    \begin{tabularx}{\textwidth}{ l X} \toprule
        \tableheadline{Category} & \tableheadline{description} \\ \midrule
        \multicolumn{2}{c}{Application software} \\ \midrule
        process & Execution unit with a protected address space.  \\
        thread & A schedulable execution path. \\
        thread group & An abstraction to logically organise threads. \\
        data & Abstraction for data units.  \\
        subprogram & Callable sequentially executable code. It represents call-return functions.  \\
        subprogram group & An abstraction to logically organise subprograms. \\ \midrule
        \multicolumn{2}{c}{Execution platform} \\ \midrule
        processor & Schedule and executes threads and virtual processors. \\
        virtual processor & Logical resource that can schedule and executes threads. It must be bound to one or more physical processor. \\
        memory & Stores code and data. \\
        bus & Interconnects processors, memory and devices. \\ \midrule
        \multicolumn{2}{c}{Composite} \\ \midrule
        system & Integrates software, hardware and other system components. \\ \midrule
        \multicolumn{2}{c}{Generic} \\ \midrule
        abstract & Define a runtime neutral component that can be refined into another component category. \\
        \bottomrule
    \end{tabularx}
    \caption[Component categories]{Component categories.}  \label{tab:categories}
\end{table}
 
Component types and implementations are defined and organised using packages; they are, essentially, libraries of component specifications that can be used in multiple architecture definitions. Packages have public and private sections to support information hiding. The public section of a package contains all the specification that will be available to other packages, while the private section can be used to hide the specific component implementation. In AADL, everything is organised in packages, an exception are property set. They are special container for user-defined properties, they act like packages and can be imported in other definition, but only properties can be defined in properties set.

To model a full architecture it is necessary to first define all the necessary component classifiers, or import the existing ones in previously defined packages. In the case of a robot, for example, it is necessary to define the physical sensors as devices and the execution platform as a combination of processors, buses and memories. On the software side, the designer could import previously defined software component as processes or define more in new packages and then import them. After this initial definition, a complete architectural description is created by integrating in a fully specified system implementation instances of the previously defined component classifiers. This hierarchy represents all the interactions between components and the architectural structure of the modelled system. These interactions cover multiple aspect of the system, they encode the communication between components through data and events, and the physical connections between them. They also capture the assignment of software to hardware (\eg, on which physical processor or processing unit a specific process will be executed). The full model of the system under analysis is obtained by instantiating this top level system implementation. This instance model can the be used to analyse operational properties of the system, ranging from syntactic compliance and basic interface data consistency to assessment of quality attributes and behaviours.

The key characteristics that make AADL suitable for our approach are the inheritance between components and the possibility to use partially defined components and interfaces that can be refined later in the design process. In practice, inheritance exists as a form of extension of existing components. A new classifier (\ie, component type and implementations) can be defined by extending an existing one; the extended classifier inherits all the characteristics of the base one: interfaces, subcomponents, properties, internal connections and modes. The extension declaration can be used to refine the new classifier by adding new elements, specifying existing elements inherited from the base classifier, restricting subcomponents to a specific mode, completing the definition of partially defined sections. Partial definition is achieved in two ways: by using abstract components or by exploiting prototypes.

The abstract component is a generic category that can be used in place of any other component type or implementation without having to specify a runtime category. A model with an abstract component cannot be instantiated, however they are extremely useful to define the initial conceptual description of the system during an iterating design process, or architecture templates and patters that can be used as reference libraries by designers. The prototypes act as placeholders for classifiers and they can be referenced anywhere a classifier would normally be referenced. The actual classifier can be specified later when referencing the parametrised component, \eg, when extending the classifier or when declaring a subcomponent. Prototypes are useful to create reference architectures or configurable product line families by providing, essentially, a parametrised classifier template that a designer can easily specify while following the structure already provided. An example is the data type exchanged between two components; the template of the component define the existence of the communication channel, but it uses a prototype for the actual type of the data. Because of the prototype, the designer needs to define a data type in order to be able to instantiate the model, but there is no restriction of the original definition of the template.

AADL is a formal declarative language described by a context-free syntax. This well-defined semantics is a key aspect of the language and a strong advantage, especially for quantitative system architectural analysis. Textual AADL is the main, more straightforward and detailed way to interact with the language, however, there are standard graphical representation that correspond to the textual definition. During the design of an AADL model, either of both representation can be used, a good strategy is to first define the skeleton of the model graphically, and then finalise the description using textual AADL. This process is supported by the Open Source AADL Tool Environment (OSATE), in this development environment a designer can easily switch between one representation of the language and the other, and any modification is propagated in all representations.

In the reminder of this section, we present more in details the component categories of AADL relevant to our work. We provide a description of the logical meaning of each category and their interactions, to better justify how used them to model a robotic architecture.

\subsection{Software components}
These categories are used to model the executable architecture of the system, they encompass functional units as processes, execution path as thread or thread groups and executable code such as functions, procedures and libraries as subprograms and subprograms groups. Moreover, the data category can be used to represent the application software artefact, some examples are data types, configuration files, internal data structures and communication messages. In addition to the semantic provided by the category itself, additional information associated to runtime (\eg, dispatch protocol and frequency of a thread) and non-runtime (\eg, source code associated to a specific subprogram) can be specified using properties.

\paragraph{Process} It represents an encapsulated execution unit; the address space, the persistent state and all internal resources are all protected and they are not accessible by external elements directly. The internal functions of the process are exposed using different kind of ports and interfaces (\ie, features): event ports can be used to trigger a behaviour or data ports for communication. Syntactically, a designer could provide access to the internal persistent state of the process, but, logically, processes usually represent protected address spaces. The process category is, basically, just a container that defines an executable entity, therefore it doesn't include an implicit definition of a thread; this means that a complete process specification has to include at least one explicitly defined thread. The allowed subcomponent categories are: thread, thread group and data. Properties can be used to specialise the runtime behaviour of a process, for example it is possible to specify the source code associated with the process, or even the actual binary that will be executed. 

\paragraph{Thread} It represents an execution path through code, that could, potentially, be executed in parallel with other similar execution paths. The executable code modelled by a thread exists within the protected address space defined by the process container. Although the name of this category suggest a direct binding between the model of a thread and a physical thread on a system, conceptually, an AADL thread is more versatile. A thread can be implemented by a single operating system thread, or represent one of multiple logical threads mapped on a physical one. A thread may also represent an active object. Logically, an AADL thread revolves around the property of being schedulable; threads can be bound to processors or virtual processor and they have multiple properties to specify their scheduling behaviour. The possible values of the \textit{Dispatch\_Protocol} property cover the most common behaviour expected by a schedulable execution path.
\begin{itemize}
\item Periodic, a repeated fixed time interval dispatch with the assumption that the execution time is shorter than the period.
\item Aperiodic, a port-based dispatch triggered by an external source, if the thread is still executing when a new dispatch arrives a queue based system is used.
\item Sporadic, the dispatch is triggered by external events on a port, but a new dispatch cannot happen before a specific interval of time.
\item Timed, thread are dispatched after a specific amount of time if no event triggers it before. Basically, it is an aperiodic dispatch with a time-out.
\item Hybrid, this dispatch method combines a periodic and aperiodic. A thread is dispatched by an external event or after a fixed amount of time.
\item Background, a thread is dispatched once and it is executed until completion.
\end{itemize}
A thread can exist only within a process or as a direct subcomponent or as part of a thread group. The possible subcomponents of a thread are: data, to capture a persistent local state, or subprogram and subprogram group, to model a local call to a functionality. The interaction between threads can happen through ports, by accessing shared data component at process-level or by calling a subprogram serviced by another thread.

\paragraph{Thread group} It can be used to organize threads within a process in a hierarchy when they are logically related or when it is necessary to create a encapsulated space with respect to the rest of the process. The unified frontier presented by threads in the same group can be used as a common interface when a designer wants to capture, at the same time, an high level functionality and the low level constituting elements. Other than thread and other thread group, the legal subcomponents are data, subprogram and subprogram group. All these subcomponents are directly accessible by the threads in the group, but are reachable by any external element only through ports.

\paragraph{Data} It can be used to model any kind of data exchanged, saved or defined in the system. Data component instances can appear in three different forms: as data subcomponents to represent persistent data (\eg, the state of an object), included in data or event data port to specify the type of data exchanged in the specific communication, as parameters declaration of subprograms. As subcomponents, a data component can have more data components, to model a record-like structure, or subprograms, to evoke the concept of a method associated to a specific data type or class. AADL is not a data modelling language, however provides enough flexibility to be used as such. The most reasonable approach is to use the data category to map all the information relevant to the model, and then exploit properties to specify a more detailed description of the data using a more suitable language.

\paragraph{Subprogram} It represents a callable unit of sequentially executable code. The subprogram type represent the signature of function, procedure or method modelled, while the subprogram implementation represent the internal functionalities. The implementation is not required to instantiate a model, however, if necessary, data components can be used to describe local variables and nested subprograms define the execution sequence. Subprograms support data access to access a shared persistent state or outgoing ports to model exceptions and errors; moreover, data components can be used to model parameters and the return value. There are two ways to model subprogram calls: by referring to the subprogram classifier, or by using a using a subprogram access feature. The first approach is used when referring directly to the subprogram (\eg, to specify the subprogram as the executable code of a thread), while the latter is used to model indirect calls (\eg, to model remote service/procedure calls or, in combination with a data component, to model an object oriented approach). Various properties of the subprogram can be used to specify the actual executable code to be used in the implementation (\eg, \textit{Source\_Name}, \textit{Source\_Text} and \textit{Source\_Language}), others are related to the calling and execution of the subprogram itself (\eg, \textit{Allowed\_Subprogram\_Call} and \textit{Compute\_Execution\_Time}).

\paragraph{Subprogram Group} It can be used to represent a collection of callable routines. For example, a subprogram group type models the API of a software library by using a series of subprogram accesses, while different subprogram group implementations can be used to model multiple implementations of the same library (\eg, different versions or implementations in different languages). The possible subcomponent of a subprogram group are: subprogram, to define the actual content of the group, subprogram group, to create a multi-level hierarchy, and data, to define a persistent state shared by all subprograms in the group.

\subsection{Execution platform components}
These categories are used to model the resources of the computer system and the elements of the external physical environment. To model the physical resources of the system, a designer can use processor,  bus and memory categories. Each of these categories represent the concept behind these physical system and not the actual object, so a processor can be a CPU, but also a processor board including operating system functionalities. In the same way, a bus component can be used to model a physical bus on a board, or a network connection such as Ethernet or CAN bus. Both these components have their virtual counterpart: a virtual processor can represent a scheduler or a virtual execution environment, while a virtual bus can model a communication protocol or a virtual channel. Memory components represent any kind of memory present in a system, form RAM to cache as well as persistent memory such as hard drives. To model sensors, actuator or physical elements of the system it is possible to use devices.

\paragraph{Processor} The definition of processor is related to the concept of thread. A processor represents the hardware and associated software that is in charge of scheduling  and executing of threads. In practice, this category can be used to model both low-level hardware of an embedded system and the high-level platforms together with operating system services, depending on nature of the model and the system. To support this, memory and bus are possible subcomponents and they can be used to define the internal function of the execution platform. To correctly instantiate the model, a processor has to be associated with a memory, it can be internal as a subcomponent or external connected via a bus. The properties available can be used to specify the runtime characteristics of the hardware (\eg, \textit{Clock\_Period}) or the physical description of the component (\eg, \textit{Hardware\_Description\_Source\_Text}).

\paragraph{Virtual processor} It represent the logical counterpart of a processor, it is a virtual resource for scheduling and executing software. It can be used to model any kind of virtualization platform (\eg, Java VMs, Docker containers, virtual environments), partitions of physical processors or hierarchies of schedulers. To instantiate a model a virtual processor has to be associated to a physical one, or as a subcomponents or by binding. Properties specific to this category are related to the binding between virtual and real processors, the others are the same of the processor category, with the exception of those used to describe the physical hardware (\eg, hardware description and clock properties).

\paragraph{Memory}  It represents any kind of storage for data and executable code. A memory category can be used to model randomly accessible physical storage (\eg, RAM and ROM), reflective memory, or permanent storage. A memory component can be used as a subcomponent of a processor to model a complete execution platform, or can exist as independent in a system to define more complex architectures or shared memories. Typically, two types of software components are bound to memories: process and data. A process has memory requirements for code, static and dynamic data, while a data component bound to a memory represent persistent data shared between different threads. Properties can be used to define the physical characteristics of the memory, such as word and total size, base address and access protocol.

\paragraph{Bus} It represents the physical connection between hardware components and the associated communication protocols. Some examples of the type of connection modelled by the bus category are PCI, CAN, Ethernet and wireless network. Another use of this category is to represent physical resources distributed to multiple physical components such as electrical power. Bus can exists as a subcomponent to any other execution platform category (\ie, processor, memory, device), however, nested buses are not permitted; only a virtual bus is accepted as subcomponent. Properties can be used to specify details about the physical connection (\eg, \textit{Transmission\_Time} and \textit{Allowed\_Message\_Size}).

\paragraph{Virtual bus} It represents a logical abstraction of a communication channel, such as a virtual partition of a physical bus, communication protocols or hierarchies of protocols by defining dependencies between multiple virtual buses. Since this category can be used to represent protocols, it can be referred in other components properties (\eg, a processor specifying \textit{Provided\_Virtual\_Bus\_Class} to specify their supported communication standards.

\paragraph{Device} It represents entities that interface with or are part of the external environment, such as sensors (\eg, cameras, laser rangefinder, GPS), actuators (\eg, motors, valves, pumps), or peripheral I/O. A device component has a dual software and hardware nature, since, as an abstraction, it can be used to model the physical component together with its driver; this means that a device support both ports and subprogram accesses to communicate with software components and bus accesses to interact with hardware components. The subcomponents available for the device are used to better describe the interaction between the external element and the system; a virtual bus can be used to specify the protocol of the communication, a bus to model the physical connection provided and a data component to capture the type of the data exchanged.

\subsection{Composite and generic components}
System and abstract component categories are not directly associated neither to software nor to hardware components, they are used to define conceptual and generic constructs. They provide to AADL the tools necessary to support modular and reusable models, by aggregating component together and by providing partially defined interfaces that can be refined during successive design phases.

\paragraph{System} It is an abstraction that represent a composite component (\ie, a container for other components). It can include software, execution platform or other system components with no restriction. This means that it is possible to create system containing only hardware components (\eg, a processor board), only software components (\eg, a software control system), a combination of software and hardware (\eg, a complete embedded system),  or a combination of all these as direct subcomponents or as contained in other systems. Even the extreme case of a system consisting only system components can be used as a generic representation of a component-based architecture. Given its nature as a container and aggregator, any type of component is an admitted subcomponent of a system. Although the aggregation defined by the system is only conceptual, it creates an actual frontier between the subcomponents inside and those outside; this means that any communication needs to go through features (\ie ports and accesses) defined on the system.

\paragraph{Abstract} It is a generic component category that can be used to declare a component type and implementation without specifying a specific category. By using this component as the only category in a system it is possible to create a conceptual component-based view of an architecture. Alternatively, by combining abstract and normal components in a system definition a designer can define a reference architecture to be specialised when necessary. Lastly, a abstract components can be used to create partially defined components that act as libraries of design patters. Abstract categories can be refined in any other category, for this reason any component (software or execution platform) is admitted as a subcomponent. The same is true for properties, since the abstract category support every possible properties. However, when an abstract component is refined to an actual category, only the properties and subcomponents admitted for that category are valid.

\subsection{Connections}


\section{AADL for robotics}
General introduction on why AADL is suitable for robotics

\subsection{Component-and-connector structure}
Lorem ipsum

\subsection{Robot operating system}
Lorem ipsum

\subsection{ROS templates}
Could be an independent subsection or part of the previous one

\section{Data Modelling}
Lorem ipsum

\subsection{ASN.1}
Lorem ipsum

\subsection{JSON}
Lorem ipsum

\subsection{JSON schema}
Lorem ipsum

%*****************************************
